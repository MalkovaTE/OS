%% -*- coding: utf-8 -*-
\documentclass[12pt,a4paper]{scrartcl} 
%\usepackage[14pt]{extsizes}
\usepackage[utf8]{inputenc}
\usepackage[english,russian]{babel}
\usepackage{indentfirst}
\usepackage{misccorr}
\usepackage{graphicx}
\usepackage{amsmath}
\usepackage[left=20mm, top=15mm, right=15mm, bottom=15mm, nohead, footskip=10mm]{geometry}
\begin{document}
\begin{titlepage}
  \begin{center}
    \large
    МИНИСТЕРСТВО ОБРАЗОВАНИЯ И НАУКИ \\РОССИЙСКОЙ ФЕДЕРАЦИИ\\
    федеральное государственное автономное образовательное учреждение высшего образования\\
    «САНКТ-ПЕТЕРБУРГСКИЙ ГОСУДАРСТВЕННЫЙ УНИВЕРСИТЕТ 
    АЭРОКОСМИЧЕСКОГО ПРИБОРОСТРОЕНИЯ»

    Кафедра 43   
    \end{center}
	\vfill
	\noindent КУРСОВОЙ ПРОЕКТ
    \normalsize{}\\
    \normalsize{ЗАЩИЩЕН С ОЦЕНКОЙ}\\
    \normalsize{РУКОВОДИТЕЛЬ}
    
    \underline{ст.преп}
    \hspace{5cm}
    \underline{\hspace{4cm}}
    \hspace{4cm}
    \underline{М.Д.Поляк}
    \vfill
    
	\begin{center}
	\normalsize{ПОЯСНИТЕЛЬНАЯ ЗАПИСКА}\\
	\normalsize{К КУРСОВОМУ ПРОЕКТУ}\\
	\vfill
	\normalsize{Мониторинг USB}\\
	\vfill
    \textsc{по дисциплине: ОПЕРАЦИОННЫЕ СИСТЕМЫ}\\
	\end{center}

	\vfill
	\noindent РАБОТУ ВЫПОЛНИЛ
	\normalsize{}\\  
	\normalsize{СТУДЕНТКА ГР.}\hspace{1cm}\underline{4331}
	\hspace{2cm}
	\underline{\hspace{3cm}}
	\hspace{3cm}
	\underline{Т.Э. Иванова.}
\vfill

\begin{center}
  Санкт-Петербург, 2016 г.
\end{center}
\end{titlepage}
\newpage
%\normalsize{Цель работы:реализовать демон для flash накопителей, осуществляющий резервное копирование данных под ОС Linux}\\
\tableofcontents % Вывод содержания
\newpage
\section{Цель работы}
	\normalsize{Цель работы: реализовать демон для flash накопителей, осуществляющий постоянный мониторинг нахождения нужного USB накопителя и записывающий информацию в лог  под ОС Linux}
\section{Задание}
	Реализовать демон для flash-накопителя, работающего через интерфейс USB, реализующий  мониторинг  USB портов на наличие нужного накопителя с правильным серийным номером (прописанном в тексте программы).  Демон производит запись в лог с указанием нахождения или ненахождения нужного накопителя в портах и точного времени проверки.
\section{Техническая документация}
\subsection{Установка}
	Склонировать репозиторий с github при помощи команды: \begin{verbatim}
	git clone https://github.com/ivanovaTE/OS
	\end{verbatim} Для работы демона должен быть установлен пакет gcc.
\subsection{Использование}
	Демон компилируется командой gcc program.c -o program и зупускается командой ./program. После запуска демон выдаст номер процесса. При необходимости его можно будет завершить вручную командой kill "номер процесса".
\section{Выводы}
	В процессе выполнения данной курсовой работы мною были получены знания и навыки, необходимые для работы с USB носителями, демонами в ОС Linux.
	\newpage
\section{Приложения}
deamon.ру:\begin{verbatim}
#include <stdio.h>
#include <stdlib.h>
#include <unistd.h>
#include <sys/types.h>
#include <sys/stat.h>
#include <string.h>
#include <time.h>

static char* path_log_file = (char*)"/home/tanya/Documents/daemon.log";
static char* serial = (char*)"36UUGDC5VEG79P2Z";
static char* dir_serial1 = (char*)"/media/tanya/EC1667C216678C80/";

char* getSerial( char *str ) 
{
    ssize_t len;
    char buf[256];
    char *p;
    char buf2[256];
    int i;
    static char comText[256];
    bzero(comText, 256);
    strcpy(comText, "");

    len = readlink(str, buf, 256);
    if (len <= 0) {
        return (char*)"-";
    }
    buf[len] = '\0';
    sprintf(buf2, "%s/%s", "/sys/block/", buf);
    for (i=0; i<6; i++) {
        p = strrchr(buf2, '/');
        *p = 0;
    }
    strcat(buf2, "/serial");

    int f = open(buf2, 0);
    if (f == -1) return (char*)"-";
    len = read(f, buf, 256);
    if (len <= 0) {
        return (char*)"-";
    }
    buf[len-1] = '\0';
    strcat(comText, buf);

    return comText;
}

int isSerial(char* ser)
{
    if (strcmp(getSerial((char*)"/sys/block/sdb"), ser) == 0
        || strcmp(getSerial((char*)"/sys/block/sdc"), ser) == 0
        || strcmp(getSerial((char*)"/sys/block/sdd"), ser) == 0
        || strcmp(getSerial((char*)"/sys/block/sdf"), ser) == 0
        || strcmp(getSerial((char*)"/sys/block/sdg"), ser) == 0)
    {
        return 1;
    }
    return 0;
}


int main(int argc, char* argv[])
{
FILE *fp= NULL;
pid_t process_id = 0;
pid_t sid = 0;
// Create child process
process_id = fork();
// Indication of fork() failure
if (process_id < 0)
{
printf("fork failed!\n");
// Return failure in exit status
exit(1);
}
// PARENT PROCESS. Need to kill it.
if (process_id > 0)
{
printf("process_id of child process %d \n", process_id);
// return success in exit status
exit(0);
}
//unmask the file mode
umask(0);
//set new session
sid = setsid();
if(sid < 0)
{
// Return failure
exit(1);
}


chdir("/");
// Close stdin. stdout and stderr
close(STDIN_FILENO);
close(STDOUT_FILENO);
close(STDERR_FILENO);
// Open a log file in write mode.
fp = fopen (path_log_file, "w+");
while (1)
{

sleep(1);
if(isSerial(serial)) {
time_t t=time(NULL);
struct tm tm= *localtime (&t);

fprintf(fp, "Correct USB in d-d-d-d n", tm.tm_mday, tm.tm_hour, tm.tm_min, tm.tm_sec);

}
else {time_t t=time(NULL);
struct tm tm= *localtime (&t);

fprintf(fp, "No USB in d-d-d-d n", tm.tm_mday, tm.tm_hour, tm.tm_min, tm.tm_sec);}
fflush(fp);

}
fclose(fp);
return (0);
}
Contact GitHub API Training Shop Blog About
© 2017 GitHub, Inc. Terms Privacy 

\end{verbatim}
	\newpage
\end{document}